\documentclass[man]{apa6}

\usepackage{amssymb,amsmath}
\usepackage{ifxetex,ifluatex}
\usepackage{fixltx2e} % provides \textsubscript
\ifnum 0\ifxetex 1\fi\ifluatex 1\fi=0 % if pdftex
  \usepackage[T1]{fontenc}
  \usepackage[utf8]{inputenc}
\else % if luatex or xelatex
  \ifxetex
    \usepackage{mathspec}
    \usepackage{xltxtra,xunicode}
  \else
    \usepackage{fontspec}
  \fi
  \defaultfontfeatures{Mapping=tex-text,Scale=MatchLowercase}
  \newcommand{\euro}{€}
\fi
% use upquote if available, for straight quotes in verbatim environments
\IfFileExists{upquote.sty}{\usepackage{upquote}}{}
% use microtype if available
\IfFileExists{microtype.sty}{\usepackage{microtype}}{}

% Table formatting
\usepackage{longtable, booktabs}
\usepackage{lscape}
% \usepackage[counterclockwise]{rotating}   % Landscape page setup for large tables
\usepackage{multirow}		% Table styling
\usepackage{tabularx}		% Control Column width
\usepackage[flushleft]{threeparttable}	% Allows for three part tables with a specified notes section
\usepackage{threeparttablex}            % Lets threeparttable work with longtable

% Create new environments so endfloat can handle them
% \newenvironment{ltable}
%   {\begin{landscape}\begin{center}\begin{threeparttable}}
%   {\end{threeparttable}\end{center}\end{landscape}}

\newenvironment{lltable}
  {\begin{landscape}\begin{center}\begin{ThreePartTable}}
  {\end{ThreePartTable}\end{center}\end{landscape}}

  \usepackage{ifthen} % Only add declarations when endfloat package is loaded
  \ifthenelse{\equal{\string man}{\string man}}{%
   \DeclareDelayedFloatFlavor{ThreePartTable}{table} % Make endfloat play with longtable
   % \DeclareDelayedFloatFlavor{ltable}{table} % Make endfloat play with lscape
   \DeclareDelayedFloatFlavor{lltable}{table} % Make endfloat play with lscape & longtable
  }{}%



% The following enables adjusting longtable caption width to table width
% Solution found at http://golatex.de/longtable-mit-caption-so-breit-wie-die-tabelle-t15767.html
\makeatletter
\newcommand\LastLTentrywidth{1em}
\newlength\longtablewidth
\setlength{\longtablewidth}{1in}
\newcommand\getlongtablewidth{%
 \begingroup
  \ifcsname LT@\roman{LT@tables}\endcsname
  \global\longtablewidth=0pt
  \renewcommand\LT@entry[2]{\global\advance\longtablewidth by ##2\relax\gdef\LastLTentrywidth{##2}}%
  \@nameuse{LT@\roman{LT@tables}}%
  \fi
\endgroup}


\ifxetex
  \usepackage[setpagesize=false, % page size defined by xetex
              unicode=false, % unicode breaks when used with xetex
              xetex]{hyperref}
\else
  \usepackage[unicode=true]{hyperref}
\fi
\hypersetup{breaklinks=true,
            pdfauthor={},
            pdftitle={TBD},
            colorlinks=true,
            citecolor=blue,
            urlcolor=blue,
            linkcolor=black,
            pdfborder={0 0 0}}
\urlstyle{same}  % don't use monospace font for urls

\setlength{\parindent}{0pt}
%\setlength{\parskip}{0pt plus 0pt minus 0pt}

\setlength{\emergencystretch}{3em}  % prevent overfull lines


% Manuscript styling
\captionsetup{font=singlespacing,justification=justified}
\usepackage{csquotes}
\usepackage{upgreek}

 % Line numbering
  \usepackage{lineno}
  \linenumbers


\usepackage{tikz} % Variable definition to generate author note

% fix for \tightlist problem in pandoc 1.14
\providecommand{\tightlist}{%
  \setlength{\itemsep}{0pt}\setlength{\parskip}{0pt}}

% Essential manuscript parts
  \title{TBD}

  \shorttitle{Entropy and Typing}


  \author{Matthew J. C. Crump\textsuperscript{1}, Walter Lai\textsuperscript{1}, \& Nicholaus Brosowsky\textsuperscript{1}}

  % \def\affdep{{"", "", ""}}%
  % \def\affcity{{"", "", ""}}%

  \affiliation{
    \vspace{0.5cm}
          \textsuperscript{1} Brooklyn College of the City University of New York  }

  \authornote{
    In this draft, listed authorship order simply indicates who is
    participating in the project
    
    Correspondence concerning this article should be addressed to Matthew J.
    C. Crump, Brooklyn College of CUNY, 2900 Bedford Avenue, Brooklyn, NY,
    11210. E-mail:
    \href{mailto:mcrump@brooklyn.cuny.edu}{\nolinkurl{mcrump@brooklyn.cuny.edu}}
  }


  \abstract{Enter abstract here. Each new line herein must be indented, like this
line.}
  \keywords{keywords \\

    \indent Word count: X
  }





\usepackage{amsthm}
\newtheorem{theorem}{Theorem}[section]
\newtheorem{lemma}{Lemma}[section]
\theoremstyle{definition}
\newtheorem{definition}{Definition}[section]
\newtheorem{corollary}{Corollary}[section]
\newtheorem{proposition}{Proposition}[section]
\theoremstyle{definition}
\newtheorem{example}{Example}[section]
\theoremstyle{definition}
\newtheorem{exercise}{Exercise}[section]
\theoremstyle{remark}
\newtheorem*{remark}{Remark}
\newtheorem*{solution}{Solution}
\begin{document}

\maketitle

\setcounter{secnumdepth}{0}



Hicks Law refers to the noticeable slowing of responses in choice
reaction experiments when the possible stimuli are less predictable.

Hicks in his experiment, \enquote{On the rate of gain of information},
tested his reaction speed on a choice reaction test. During each trial
he had to determine which of n lights was turned on. He manipulated the
set size n, and found that the greater the number of alternatives the
slower his response. His data also suggested that reaction speed is
linearly related to the predictability of responses; Trials in which
certain responses were disproportionately more likely ellicited faster
response times than those in which each response was equally likely.

Hick's law at larger set sizes however does not have consistent
experimental support.

Conrad (1962) in an experiment which required subjects to name 320
nonsense syllables within trials of variable number of distinct
syllables (number of alternatives) found that the response time was
linearly correlated with the logarithm of number of alternatives.
However, Pierce and Karlin in 1957 in an experiment which required
subjects to read pages of words as quickly as possible, reported no
differences in response times between trials with set sizes that ranged
from 4 to 256 words.\\
Proctor and Schneider in a review of studies testing Hick's Law
attributes the inconsistency in large set size experiments to
differences in skill level or practice between subjects and the
arbitrariness of Stimulus Response coding/mappings when creating more
number of alternatives.

Our experiment sought to test Hick's law on typing speeds.

In this study, we used typing data from 346 typists to analyze the
effect of letter uncertainty on response times. In essence, every letter
typed represents one choice reaction test trial from Hick's experiment.
In Hick's experiment, the probability of each light turning on was
manipulated to change the H value. This is analagous to the probability
of certain letters appearing at a specific position in a word. Peter
Norvig analyzed words from numerous texts and has determined the
probability for each letter appearing at a specific position in 2 to 9
letter words.

For example, the first letter of a two letter long word is has a 9.44\%
chance to be an \enquote{a}, 14.9\% chance to be a \enquote{t}, and a
25\% chance to be an \enquote{i}. The third letter of a five letter long
word is has a 13.5\% chance as an \enquote{e}, 4.91\% chance as a
\enquote{t}, and a 11.4 \% chance as a \enquote{a}. The H indices for
each letter position - word length pair can be calculated. The H for the
1st Letter, 2-letter-long word (abbreviated 1:2) is 2.85. The H for the
3rd Letter, 5-letter-long word (abbreviated 3:5) is 3.94. The
differences in H tell us that the first letter of a two letter long word
has lower uncertainty ie. it is more predictable than the third letter
of a five letter long word.

This combination of letter probabilities creates an H value for each
unique pair of letter position and word length.

One potential problem that can hinder our experiment's ability to test
the Hick's Law is the fact that our typists vary in terms of their skill
level. This means that differences in RT between subjects will generate
noise that may hide the H effect on RTs. An advantage of testing Hick's
Law with this data set is that the stimulus response coding is
non-arbitrary. The letters that typists see on the screen correspond
exactly to the letters on their keyboards.

\section{Methods}\label{methods}

The uncertainty for each letter position in 1-9 letter long words was
calculated using Norvig's ngrams1.xlsx from his
website:\url{http://norvig.com/mayzner.html.and} using the equation H =
N sigma (PlnP) where N is the number of alternatives (26 possible
letters) and P is probability of a given letter occurring at a specific
letter position in a word of specific length.\\
As an example, I demonstrate calculation of the H value of the 2nd
letter of 3 letter words: H = N sigma (PlnP) Total number of 3 letter
words observed by Norvig: 1.52* 10\^{}11 three letter words
P(\enquote{a})2 position = 3.1\emph{10\^{}10 three letter words with
\enquote{a} as 2nd letter/1.52} 10\^{}11 three letter words
P(\enquote{b})2 position = 3.64\emph{10\^{}9 three letter words with
\enquote{b} as 2nd letter/1.52} 10\^{}11 three letter words : :
P(\enquote{z})2 position = 3.00\emph{10\^{}7 three letter words with
\enquote{z} as 2nd letter/1.52} 10\^{}11 three letter words H = 26
possible alternatives * sigma (as letter from \enquote{a} to
\enquote{z}) P\enquote{letter}2position *ln P\enquote{letter}2position H
= 2.85 bits\\
The relationship between uncertainty and letter position is visualized
in Graph \#. We observe a trend of increasing uncertainty near the
middle letter position of most words.

\subsection{Data analysis}\label{data-analysis}

We used R (Version 3.4.3; R Core Team, 2017) and the R-package
\emph{papaja} (Version 0.1.0.9709; Aust \& Barth, 2018) for all our
analyses.

Interkeystroke intervals were recorded for each of the 4000 letters
typed per typist. The length of the interkeystroke interval immediately
preceding a given letter was considered the response time for that
letter. Each cell contained the average response time for all the
letters that shared a similar letter position in words of similar length
for each subject (Subject\emph{Letter Position } Word Length).\\
Outlier response times were removed on a cell by cell basis using the
non-recursive with moving criterion procedure of Van Selst Jolicoeur
(1994). This procedure eliminated 4.9\% of letters from further
analysis. Uncertainty Effects on Response Times

\section{Results}\label{results}

We observed a general trend of increased response times in the middle
letter positions of words. The increases in response time visually
correspond to the increases in uncertainty at the middle letter
positions of words with of 1 to 9 letters.

The correlation between response times and uncertainty was computed for
each subject for each letter position for words of 1 to 9 letters. The r
squared values were averaged and a mean r squared of 0.118 was obtained.

\section{Discussion}\label{discussion}

The takeaway from this experiment was that a typist types a letter
faster when the letters commonly found at the given letter position are
few. These findings are consistent with Hick's Law. As the alternatives
become equiprobable, as it is observed in the middle of words, the
number of bits required to process the choices increases.

\section{Conclusion}\label{conclusion}

Future studies should investigate the role of probability of repetition
in regulating response times.\\
Kornblum in 1969 found that with constant H, trials that had higher
chances of sequentially repeating stimuli experienced faster response
times.

\newpage

\section{References}\label{references}

\begingroup
\setlength{\parindent}{-0.5in} \setlength{\leftskip}{0.5in}

\hypertarget{refs}{}
\hypertarget{ref-R-papaja}{}
Aust, F., \& Barth, M. (2018). \emph{papaja: Create APA manuscripts with
R Markdown}. Retrieved from \url{https://github.com/crsh/papaja}

\hypertarget{ref-R-base}{}
R Core Team. (2017). \emph{R: A language and environment for statistical
computing}. Vienna, Austria: R Foundation for Statistical Computing.
Retrieved from \url{https://www.R-project.org/}

\endgroup






\end{document}
